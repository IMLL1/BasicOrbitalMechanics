\documentclass{article}
\usepackage{indentfirst}
\usepackage{graphicx}
\graphicspath{{images/}}
\usepackage{amsmath}
\usepackage{amssymb}
\usepackage{cancel}
\usepackage{esvect}
\usepackage{mathtools,leftidx}
\usepackage[dvipsnames]{xcolor}
\usepackage{tikz}
\usepackage{tikz-3dplot}
\usetikzlibrary{arrows.meta}
\usetikzlibrary{arrows}
\usetikzlibrary{decorations.pathreplacing}
\usetikzlibrary{decorations.markings}
\usepackage{float}
\usepackage{pgffor}
\usepackage[pdfencoding=auto, psdextra]{hyperref}
\hypersetup{
    colorlinks = true, %Colours links instead of ugly boxes
    urlcolor = blue, %Colour for external hyperlinks
    linkcolor = blue, %Colour of internal links
}
\usepackage[a4paper, left=2cm,right=2cm, top=4cm, bottom=4cm]{geometry}
\usepackage{comment}

\newcommand{\st}[2]{{#1}_\text{#2}} % latin subscript

\usepackage{subfiles} % Best loaded last in the preamble
\allowdisplaybreaks

\title{Elementary Orbital Mechanics}
\author{(Removed for Internet Anonymity)\\Purdue University, School of Aeronautics and Astronautics}
\date{}

\begin{document}

\maketitle

There are many publications that purportedly cover the basics of astrodynamics and orbital mechanics, and many even claim to be catered to an uninitiated audience. However, many of them take for granted some prior knowledge of orbital mechanics, and neglect the ever-important step of ensuring that the reader can follow \textit{every} step of the process. This document intends to rectify this by covering the basics of orbital dynamics, from derivations of orbital motion to complex orbital maneuvers, while ensuring the reader can follow every step. This means algebraic steps are shown, motivations are discussed (where possible) instead of completing procedures seemingly out of the blue, and conclusions are summarized. It is reccomended that the reader view the appendix first, to get a sense for some terminology that is used throughout this document. Note that this document examines exclusively the restricted two-body problem; the impact of an orbiting body on the orbited body is neglected entirely, meaning that the orbited body has no inertial acceleration. While much of this was derived without inspiration, citations are provided for sections where ideas were drawn from specific publications. This was written after having skimmed Fundamentals of Astrodynamics \cite{Bate_Mueller_White_1971}, so even where no citations are provided, complete intellectual independence cannot be claimed.

\pagebreak
\tableofcontents

\pagebreak
\section{Definitions of Motion}\label{Sec:Motion}
\subfile{sections/motion}

\pagebreak
\section{Orbit Equation}\label{Sec:Orbit Equation}
\subfile{sections/orbitEquation}

\pagebreak
\section{Proofs}\label{sec:Proofs}
\subfile{sections/proofs}

\pagebreak
\section{Orbit Geometry}\label{sec:Orbit Geometry}
\subfile{sections/orbitGeometry}

\pagebreak
\section{Physical Orbital Parameters from Geometry}\label{sec:Orbital Parameters from Geometry}
\subfile{sections/physicalFromGeometry}

\pagebreak
\section{Geometry from Physical Parameters}\label{sec:Geometry from Physical Parameters}
\subfile{sections/geometryFromPhysical}

\pagebreak
\section{Analysis of Unbound Trajectories}\label{sec:Special Trajectories}
\subfile{sections/unboundTrajectories}

\pagebreak
\section{Orbital Manuevers Basics}\label{sec:Manuevers Basics}
\subfile{sections/maneuversBasics}

\pagebreak
\section{Single-Body Maneuvers}\label{sec:Manuevers}
\subfile{sections/basicManeuvers}

\pagebreak
\section{Multi-Body Maneuvers}\label{sec:Multibody Maneuevers}
\subfile{sections/multibodyManeuvers}

\pagebreak
\section{Special Orbits}\label{sec:Special Orbits}
\subfile{sections/specialOrbits}

\pagebreak
\section{Brief Note on n-Body Motion}\label{sec:N Body Motion}
\subfile{sections/nBodyMotion}

\pagebreak
\section{Appendix}\label{sec:Appendix}
\subfile{sections/appendix}

% prove that the frame can be doing whatever tf it wants

\bibliographystyle{plain} % We choose the "plain" reference style
\bibliography{refs} % Entries are in the refs.bib file

\end{document}