\documentclass[../basicOrbitalDynamics.tex]{subfiles}
\graphicspath{{\subfix{../images/}}}
\begin{document}

\bigskip\bigskip
\subsection{Gravity Assist}

One maneuver that is used frequently in movies is the gravity assist (sometimes in pop culture referred to as a slingshot maneuver). Intuitively, this may seem to violate what we have learned so far. As a satellite approaches a gravitational source, its velocity increases, trading kinetic energy for potential energy. However, as the satellite departs from the gravitational source, the tradeoff will occur in reverse, with no kinetic energy being gained. This would seem to imply that a spacecraft cannot gain speed just by flinging around a planet.

A flyby will take the shape of a hyperbolic trajectory, entering at some angle relative to reference direction and leaving at some other angle. From Equation  \eqref{Vis-Viva Equation}, the speed is only distance-dependant. A spacecraft's velocity at some distance $r$ from the gravity source will not change whether the spacecraft is approaching the gravity source or leaving it. Of course, the existence of a gravity assist maneuver means that there must be some error in this logic.

For the sake of this exploration, a gravity assist from Jupiter in a solar orbit will be considered.

\begin{figure}[H]
    \centering
    \begin{tikzpicture}[>=latex]
        \def\ecc{1.2}
        \def\arcRad{1}
        \def\m{\fpeval{sqrt(\ecc^2-1)}}
        \def\endT{2}

        \coordinate (p1) at (-{cosh(\endT)+\ecc},{-\m*sinh(\endT)}) {};
        \coordinate (p2) at (-{cosh(\endT)+\ecc},{\m*sinh(\endT)}) {};

        \draw[scale=1,domain=-\endT:\endT,smooth,variable=\t, thin, densely dotted] plot ({-cosh(\t)+\ecc},{-\m*sinh(\t)});
        \filldraw[brown] (0,0) circle (2pt);

        \filldraw[gray] (p1) circle (1pt);
        \filldraw[gray] (\ecc-1,0) circle (1pt);
        \filldraw[gray] (p2) circle (1pt);

        \draw[gray, ->] (p1) node[below right] {$v_1$} -- +(0.5,0.5*\m);
        \draw[gray, ->] (\ecc-1,0) -- +(0,1) node[right] {$v_2>v_1$};
        \draw[gray, ->] (p2) node[above right] {$v_3=v_1$} -- +(-0.5,0.5*\m);
    \end{tikzpicture}
    \caption{A hyperbolic trajectory of a satellite (gray) past Jupiter (brown)}\label{fig:Jupiter Flyby}
\end{figure}

The critical piece that is missing from this understanding so far is the relative nature of velocity. Equations so far have all been in terms of the body being orbited. It is indeed true that, in the above image, $|v_3|=|v_1|$ relative to Jupiter. However, this is not the complete picture. When in a hyperbolic trajectory past Jupiter, the spacecraft's speed \textit{relative to Jupiter} is dependant only on its position. However, relative to the sun this may not be true.

Figure  \ref{fig:Jupiter Flyby} can be redrawn to include velocities relative to the sun. It will be decided that Jupiter is moving to the left sun. The sun will not be shown. Velocities relative to the sun will be drawn in red, while velocities relative to Jupiter will be drawn in brown. Subscripts $j$ and $s$ mean relative to Jupiter and to the sun, respectively.

\begin{figure}[H]
    \centering
    \begin{tikzpicture}[>=latex]
        \def\ecc{1.2}
        \def\arcRad{1}
        \def\m{\fpeval{sqrt(\ecc^2-1)}}
        \def\endT{2}
        \def\vJ{1.5}

        \coordinate (p1) at (-{cosh(\endT)+\ecc},{-\m*sinh(\endT)}) {};
        \coordinate (p2) at (-{cosh(\endT)+\ecc},{\m*sinh(\endT)}) {};
        \coordinate (vJup) at (-\vJ,0) {};

        \draw[scale=1,domain=-\endT:\endT,smooth,variable=\t, thin, densely dotted] plot ({-cosh(\t)+\ecc},{-\m*sinh(\t)});


        \draw[red, ->] (0,0) -- (vJup) node[left] {$v_{\text{Jupiter},s}$};
        \draw[brown, ->] (p1) -- +(0.5,0.5*\m) node[midway, below right] {$v_{\text{enter},j}$};
        \draw[brown, ->] (p2) -- +(-0.5,0.5*\m) node[midway, right] {$v_{\text{exit},j}$};

        \draw[lightgray, thin, densely dashed, -{Latex[open]}] (p1)++(0.5,0.5*\m)--+(vJup) node[midway, above] {$v_{\text{Jupiter},s}$};
        \draw[lightgray, thin, densely dashed, -{Latex[open]}] (p2)++(-0.5,0.5*\m)--+(vJup) node[midway, above] {$v_{\text{Jupiter},s}$};

        \draw[red, ->] (p1) -- + (0.5-\vJ,0.5*\m) node[midway, below] {$v_{\text{enter},s}$};
        \draw[red, ->] (p2) --+ (-0.5-\vJ,0.5*\m) node[midway, below] {$v_{\text{exit},s}$};


        \filldraw[gray] (p1) circle (1pt);
        \filldraw[gray] (p2) circle (1pt);
        \filldraw[brown] (0,0) circle (2pt);
    \end{tikzpicture}
    \caption{Jupiter Flyby With Relative Velocities}\label{fig:Jupiter Flyby Rel Velocities}
\end{figure}

Note that Jupiter's velocity relative to the sun is also shown at the spacecraft's location in order to make the vector addition visually intuitive. As can be seen in Figure \ref{fig:Jupiter Flyby Rel Velocities}, the spacecraft's velocity relative to the sun is changed by Jupiter.

While this may seem to violate conservation of energy and momentum, it does not. Any momentum or energy gained by the spacecraft will be lost by the planet that's providing the gravity assist. While this does have an effect on the orbit of the planet that's providing a gravity assist (Jupiter, in this case), this effect is negligible. Jupiter has a mass on the order of 10\textsuperscript{27} kilograms, while most satellites will have masses between 10\textsuperscript{2} and 10\textsuperscript{4} kilograms. This means that an energy transfer that tremendously impacts the velocity of a satellite will have a negligible effect on Jupiter.

To illustrate the varying effects that gravity assists can have, a few possible assists will be drawn.

\begin{figure}[H]
    \centering
    \begin{tikzpicture}[>=latex]
        \def\ecc{1.2}
        \def\arcRad{1}
        \def\m{\fpeval{sqrt(\ecc^2-1)}}
        \def\endT{2}
        \def\vJ{2}

        \coordinate (p1) at ({cosh(\endT)-\ecc},{-\m*sinh(\endT)}) {};
        \coordinate (p2) at ({cosh(\endT)-\ecc},{\m*sinh(\endT)}) {};
        \coordinate (vJup) at (-\vJ,0) {};

        \draw[scale=1,domain=-\endT:\endT,smooth,variable=\t, thin, densely dotted] plot ({cosh(\t)-\ecc},{-\m*sinh(\t)});


        \draw[brown, ->] (p1) -- +(-0.5,0.5*\m) node[midway, above right] {$v_{\text{enter},j}$};
        \draw[brown, ->] (p2) -- +(0.5,0.5*\m) node[midway, below right] {$v_{\text{exit},j}$};

        \draw[lightgray, thin, densely dashed, -{Latex[open]}] (p1)++(-0.5,0.5*\m)--+(vJup) node[midway, above] {$v_{\text{Jupiter},s}$};
        \draw[lightgray, thin, densely dashed, -{Latex[open]}] (p2)++(0.5,0.5*\m)--+(vJup) node[midway, above] {$v_{\text{Jupiter},s}$};

        \draw[red, ->] (p1) -- + (-0.5-\vJ,0.5*\m) node[midway, below] {$v_{\text{enter},s}$};
        \draw[red, ->] (p2) --+ (0.5-\vJ,0.5*\m) node[midway, below] {$v_{\text{exit},s}$};


        \filldraw[gray] (p1) circle (1pt);
        \filldraw[gray] (p2) circle (1pt);
        \filldraw[brown] (0,0) circle (2pt);
    \end{tikzpicture}
    \caption{Gravity Assist In Front of Planet}\label{fig:Gravity Assist Ahead of Planet}
\end{figure}

From Figures \ref{fig:Jupiter Flyby Rel Velocities} and \ref{fig:Gravity Assist Ahead of Planet}, it is clear that passing behind the planet acts to accelerate the spacecraft, while passing behind the planet decelerates the spacecraft by the same amount. Note that, while these orbits both start ``below'' Jupiter and end ``above'' it, the effect is the exact same if the direction of the flyby is reversed. Visualization of this effect is left as an exercise to the reader. Next, the effect of passing ``above'' or ``below'' the planet will be investigated.

\begin{figure}[H]
    \centering
    \begin{tikzpicture}[>=latex]
        \def\ecc{1.2}
        \def\arcRad{1}
        \def\m{\fpeval{sqrt(\ecc^2-1)}}
        \def\endT{2}
        \def\vJ{2}

        \coordinate (p1) at ({-\m*sinh(\endT)},-{cosh(\endT)+\ecc}) {};
        \coordinate (p2) at ({\m*sinh(\endT)}, -{cosh(\endT)+\ecc}) {};
        \coordinate (vJup) at (-\vJ,0) {};

        \draw[scale=1,domain=-\endT:\endT,smooth,variable=\t, thin, densely dotted] plot ({-\m*sinh(\t)},{-cosh(\t)+\ecc});

        \draw[red, ->] (0,0) -- (vJup) node[left] {$v_{\text{Jupiter},s}$};
        \draw[brown, ->] (p1) -- +(0.5*\m, 0.5) node[midway, below right] {$v_{\text{enter},j}$};
        \draw[brown, ->] (p2) -- +(0.5*\m, -0.5) node[midway, right] {$v_{\text{exit},j}$};

        \draw[lightgray, thin, densely dashed, -{Latex[open]}] (p1)++(0.5*\m, 0.5)--+(vJup) node[midway, above] {$v_{\text{Jupiter},s}$};
        \draw[lightgray, thin, densely dashed, -{Latex[open]}] (p2)++(0.5*\m, -0.5)--+(vJup) node[midway, below] {$v_{\text{Jupiter},s}$};

        \draw[red, ->] (p1) -- + (0.5*\m-\vJ,0.5) node[midway, below] {$v_{\text{enter},s}$};
        \draw[red, ->] (p2) --+ (0.5*\m-\vJ,-0.5) node[midway, above] {$v_{\text{exit},s}$};

        \filldraw[gray] (p1) circle (1pt);
        \filldraw[gray] (p2) circle (1pt);
        \filldraw[brown] (0,0) circle (2pt);

    \end{tikzpicture}
    \caption{Radial Gravity Assists}\label{fig:Gravity assists above/below}
\end{figure}

In Figure \ref{fig:Gravity assists above/below}, the magnitude of velocity of the satellite did not change; instead, its direction changed. The planet has the net effect of redirecting the spacecraft's velocity with respect to the planet that is giving the gravity assist.

In Figures \ref{fig:Jupiter Flyby}, \ref{fig:Jupiter Flyby Rel Velocities}, \ref{fig:Gravity Assist Ahead of Planet}, and \ref{fig:Gravity assists above/below}, the hyperbolic flyby had the same eccentricity (precisely 1.2). However, the direction and eccentricity of the flyby can be fine-tuned to ensure the desired effect on the spacecraft's orbit.

\subsubsection{Mathematical Analysis of Gravity Assists}

Because a hyperbolic trajectory occurs above escape velocity (See Equation \eqref{Orbit Shape Physical}), the speed that the spacecraft approaches (relative to the planet) in the limit can be determined.

\begin{align*}
    v_\infty & = \lim_{r\rightarrow\infty}v                                                \\
             & = \lim_{r\rightarrow\infty}\left(\sqrt{\frac{2\mu}{r}-\frac{\mu}{a}}\right) \\
             & = \sqrt{\frac{\mu}{-a}}
\end{align*}

Using this, the semi-major axis can be found
\begin{align*}
    v_\infty   & = \sqrt{\frac{\mu}{-a}}   \\
    v_\infty^2 & = \frac{\mu}{-a}          \\
    a          & = \frac{-\mu}{v_\infty^2} \\
\end{align*}

The semi-major axis is uniquely determined by the limiting velocity of the spacecraft, and is unaffected by the entry/exit angle $\st{\theta}{hyp}$. A more useful way of phrasing this is that the entry/exit angle is unaffected by the entry velocity.

Because $a$ is not determined by the chosen hyperbolic trajectory, from Equation \eqref{Periapsis Radius Geometric} the eccentricity can be determined by the periapsis radius. Solving Equation \eqref{Periapsis Radius Geometric} for the eccentricity,

\begin{align*}
    \st{r}{pe}                       & =a(1-e)                       \\
    \st{r}{pe}                       & =\frac{-\mu}{v_\infty^2}(1-e) \\
    \st{r}{pe}                       & =\frac{\mu}{v_\infty^2}(e-1)  \\
    \frac{\st{r}{pe}v_\infty^2}{\mu} & =e-1                          \\
\end{align*}
\begin{equation}\label{Hyperbolic Flyby Eccentricity}
    e=\frac{\st{r}{pe}v_\infty^2}{\mu}+1
\end{equation}

Recall from Equation \eqref{Hyperbolic entry/exit angle} that $\st{\theta}{hyp}$ is determined uniquely by $e$.

\begin{align*}
    \st{\theta}{hyp} & =\arctan\left(\sqrt{e^2-1}\right)                                                \\
                      & =\arctan\left(\sqrt{\left(\frac{\st{r}{pe}v_\infty^2}{\mu}+1\right)^2-1}\right) \\
\end{align*}

\begin{equation}\label{Hyperbola angle in terms of pe}
    \st{\theta}{hyp}=\arctan\left(\sqrt{\left(\frac{\st{r}{pe}v_\infty^2}{\mu}+1\right)^2-1}\right)\\
\end{equation}

With Equation \eqref{Hyperbola angle in terms of pe}, we now have the tools to determine the effect of a gravity assist on a spacecraft's velocity relative to the sun. For the sake of simplicity, the argument of $\arctan$ in Equation \eqref{Hyperbola angle in terms of pe} will be referred to as simply $\st{m}{hyp}$

The net change in a spacecraft's velocity vector relative to the sun can be expressed as follows. Subscripts $s$ and $p$ will be used to denote velocity relative to a planet versus that relative to the sun. The maneuvering basis vectors are relative to $\vv{v}_{\text{enter},p}$. The exit velocity vector will be rotated by $2\st{\theta}{hyp}$, and then reversed in direction (See figure \ref{fig:Hyperbolic Trajectory}). This has the net effect of a rotation of $\pi-2\st{\theta}{hyp}$ Recall also that $\hat{m}_r$ points \textit{out} of the orbit.

\begin{align*}
    \Delta\vv{v}_s & = \Delta\vv{v}_p                                                                                             \\
                   & = \vv{v}_{\text{exit},p}-\vv{v}_{\text{enter},p}                                                             \\
                   & = v_\infty\hat{v}_{\text{exit},p}-v_\infty\hat{v}_{\text{enter},p}                                           \\
                   & = v_\infty(\cos(\pi-2\st{\theta}{hyp})\hat{m}_v-\sin(\pi-2\st{\theta}{hyp})(-\hat{m}_r))-v_\infty\hat{m}_v \\
                   & = v_\infty(\cos(\pi-2\st{\theta}{hyp})\hat{m}_v+\sin(\pi-2\st{\theta}{hyp})\hat{m}_r)-v_\infty\hat{m}_v    \\
                   & = v_\infty(-\cos(2\st{\theta}{hyp})\hat{m}_v+\sin(2\st{\theta}{hyp})\hat{m}_r)-v_\infty\hat{m}_v           \\
                   & = v_\infty((-1-\cos(2\st{\theta}{hyp}))\hat{m}_v+\sin(2\st{\theta}{hyp})\hat{m}_r)                         \\
                   & = v_\infty(-(1+\cos(2\st{\theta}{hyp}))\hat{m}_v+\sin(2\st{\theta}{hyp})\hat{m}_r)                         \\
                   & = v_\infty(-(1+\cos(2\arctan(\st{m}{hyp})))\hat{m}_v+\sin(2\arctan(\st{m}{hyp}))\hat{m}_r)                 \\
\end{align*}

It can be shown from trigonometry that
\[\cos(2\arctan(\theta))=\frac{1-\theta^2}{1+\theta^2}\qquad\text{and}\qquad\sin(2\arctan(\theta))=\frac{2\theta}{1+\theta^2}\]

%\left(\sqrt{\left(\frac{\st{r}{pe}v_\infty^2}{\mu}+1\right)^2-1}\right)

\begin{align*}
    \Delta\vv{v}_s & = v_\infty(-(1+\cos(2\arctan(\st{m}{hyp})))\hat{m}_v+\sin(2\arctan(\st{m}{hyp}))\hat{m}_r)                                                                                                                                                                                                               \\
                   & = v_\infty\left(-\left(1+\frac{1-\st{m}{hyp}^2}{1+\st{m}{hyp}^2}\right)\hat{m}_v+\frac{2\st{m}{hyp}}{1+\st{m}{hyp}^2}\hat{m}_r\right)                                                                                                                                                                  \\
                   & = v_\infty\left(-\left(1+\frac{1-\left(\frac{\st{r}{pe}v_\infty^2}{\mu}+1\right)^2+1}{1+\left(\frac{\st{r}{pe}v_\infty^2}{\mu}+1\right)^2-1}\right)\hat{m}_v+\frac{2\sqrt{\left(\frac{\st{r}{pe}v_\infty^2}{\mu}+1\right)^2-1}}{1+\left(\frac{\st{r}{pe}v_\infty^2}{\mu}+1\right)^2-1}\hat{m}_r\right) \\
                   & = v_\infty\left(-\left(\frac{2}{\left(\frac{\st{r}{pe}v_\infty^2}{\mu}+1\right)^2}\right)\hat{m}_v+\frac{2\sqrt{\left(\frac{\st{r}{pe}v_\infty^2}{\mu}+1\right)^2-1}}{\left(\frac{\st{r}{pe}v_\infty^2}{\mu}+1\right)^2}\hat{m}_r\right)                                                                \\
                   & = v_\infty\left(\frac{-2}{\left(\frac{\st{r}{pe}v_\infty^2}{\mu}+1\right)^2}\hat{m}_v+\frac{2\sqrt{\left(\frac{\st{r}{pe}v_\infty^2}{\mu}+1\right)^2-1}}{\left(\frac{\st{r}{pe}v_\infty^2}{\mu}+1\right)^2}\hat{m}_r\right)                                                                             \\
\end{align*}

While this may not intuitively seem that useful (and, frankly, it isn't), the magnitude proves to be much simpler.

\begin{align*}
    |\Delta\vv{v}_s| & = v_\infty\left|\frac{-2}{\left(\frac{\st{r}{pe}v_\infty^2}{\mu}+1\right)^2}\hat{m}_v+\frac{2\sqrt{\left(\frac{\st{r}{pe}v_\infty^2}{\mu}+1\right)^2-1}}{\left(\frac{\st{r}{pe}v_\infty^2}{\mu}+1\right)^2}\hat{m}_r\right|       \\
                     & = v_\infty\sqrt{\left(\frac{-2}{\left(\frac{\st{r}{pe}v_\infty^2}{\mu}+1\right)^2}\right)^2+\left(\frac{2\sqrt{\left(\frac{\st{r}{pe}v_\infty^2}{\mu}+1\right)^2-1}}{\left(\frac{\st{r}{pe}v_\infty^2}{\mu}+1\right)^2}\right)^2} \\
                     & = v_\infty\sqrt{\frac{4}{\left(\frac{\st{r}{pe}v_\infty^2}{\mu}+1\right)^4}+\frac{4\left(\left(\frac{\st{r}{pe}v_\infty^2}{\mu}+1\right)^2-1\right)}{\left(\frac{\st{r}{pe}v_\infty^2}{\mu}+1\right)^4}}                          \\
                     & = 2v_\infty\sqrt{\frac{1+\left(\frac{\st{r}{pe}v_\infty^2}{\mu}+1\right)^2-1}{\left(\frac{\st{r}{pe}v_\infty^2}{\mu}+1\right)^4}}                                                                                                  \\
                     & = 2v_\infty\sqrt{\frac{\left(\frac{\st{r}{pe}v_\infty^2}{\mu}+1\right)^2}{\left(\frac{\st{r}{pe}v_\infty^2}{\mu}+1\right)^4}}                                                                                                      \\
                     & = 2v_\infty\sqrt{\frac{1}{\left(\frac{\st{r}{pe}v_\infty^2}{\mu}+1\right)^2}}                                                                                                                                                       \\
                     & = \frac{2v_\infty}{\frac{\st{r}{pe}v_\infty^2}{\mu}+1}                                                                                                                                                                              \\
\end{align*}

The magnitude of $\Delta V$ from a gravity assist is
\begin{equation}\label{Gravity Assist Delta V}
    \st{\Delta V}{assist}=\frac{2\mu_p v_\infty}{\st{r}{pe}v_\infty^2+\mu_p}\\
\end{equation}

Where $\mu_p$ refers to the standard gravitational parameter of the planet that is giving a gravity assist.

The direction of this gravity assist is the direction of the change relative velocity of the planet. From Figures \ref{fig:Jupiter Flyby Rel Velocities}, \ref{fig:Gravity Assist Ahead of Planet}, and \ref{fig:Gravity assists above/below}, the direction of change in velocity is the vector facing from the periapsis of the hyperbola to the planet.

Equation \eqref{Gravity Assist Delta V} can be optimized to find the peak $\Delta V$. It is clear that any decrease in $\st{r}{pe}$ will increase the $\Delta V$, so the optimization will be for peak $v_\infty$. This will be done by differentiating \eqref{Gravity Assist Delta V} with respect to $v_\infty$, and setting the derivative to zero.

\begin{align*}
    \frac{d}{dv_\infty}\st{\Delta V}{assist} & =\frac{d}{dv_\infty}\Delta\left(\frac{2\mu_p v_\infty}{\st{r}{pe}v_\infty^2+\mu_p}\right)                                                         \\
                                              & =\frac{d}{dv_\infty}\left(\frac{2\mu_p v_\infty}{\st{r}{pe}v_\infty^2+\mu_p}\right)                                                               \\
                                              & =2\mu_p\frac{d}{dv_\infty}\left(v_\infty(\st{r}{pe}v_\infty^2+\mu_p)^{-1}\right)                                                                  \\
                                              & =2\mu_p\left(v_\infty\frac{d}{dv_\infty}(\st{r}{pe}v_\infty^2+\mu_p)^{-1}+(\st{r}{pe}v_\infty^2+\mu_p)^{-1}\frac{d}{dv_\infty}v_\infty\right)    \\
                                              & =2\mu_p\left(-v_\infty(\st{r}{pe}v_\infty^2+\mu_p)^{-2}\frac{d}{dv_\infty}(\st{r}{pe}v_\infty^2+\mu_p)+(\st{r}{pe}v_\infty^2+\mu_p)^{-1}\right) \\
                                              & =2\mu_p\left(-v_\infty(\st{r}{pe}v_\infty^2+\mu_p)^{-2}(2\st{r}{pe}v_\infty)+(\st{r}{pe}v_\infty^2+\mu_p)^{-1}\right)                           \\
                                              & =2\mu_p\left(\frac{-2\st{r}{pe}v_\infty^2}{(\st{r}{pe}v_\infty^2+\mu_p)^2}+\frac{1}{\st{r}{pe}v_\infty^2+\mu_p}\right)                          \\
                                              & =2\mu_p\left(\frac{-2\st{r}{pe}v_\infty^2+\st{r}{pe}v_\infty^2+\mu_p}{(\st{r}{pe}v_\infty^2+\mu_p)^2}\right)                                    \\
                                              & =2\mu_p\left(\frac{-\st{r}{pe}v_\infty^2+\mu_p}{(\st{r}{pe}v_\infty^2+\mu_p)^2}\right)                                                           \\
\end{align*}

Now, this will be set to zero to find $v_\infty$ for maximum $\Delta V$
\begin{align*}
    2\mu_p\left(\frac{-\st{r}{pe}v_\infty^2+\mu_p}{(\st{r}{pe}v_\infty^2+\mu_p)^2}\right) & = 0                                \\
    \frac{-\st{r}{pe}v_\infty^2+\mu_p}{(\st{r}{pe}v_\infty^2+\mu_p)^2}                    & = 0                                \\
    -\st{r}{pe}v_\infty^2+\mu_p                                                            & = 0                                \\
    \st{r}{pe}v_\infty^2                                                                   & = \mu_p                            \\
    v_\infty^2                                                                              & = \frac{\mu_p}{\st{r}{pe}}        \\
    v_\infty                                                                                & = \sqrt{\frac{\mu_p}{\st{r}{pe}}} \\
\end{align*}

Notice that this is equal to the velocity for a circular orbit at periapsis.
\begin{equation}\label{Gravity Assist Best Vinfty}
    v_{\infty,\text{best}} = \sqrt{\frac{\mu_p}{\st{r}{pe}}} = \st{v}{circ@pe}
\end{equation}

This can be plugged into Equation \eqref{Gravity Assist Delta V} to find the optimal $\Delta V$ for a gravity assist.
\begin{align*}
    \st{\Delta V}{assist} & =\frac{2\mu_p v_\infty}{\st{r}{pe}v_\infty^2+\mu_p}                                                              \\
                           & =\frac{2\mu_p \sqrt{\frac{\mu_p}{\st{r}{pe}}}}{\st{r}{pe}\left(\sqrt{\frac{\mu_p}{\st{r}{pe}}}\right)^2+\mu_p} \\
                           & =\frac{2\mu_p \sqrt{\frac{\mu_p}{\st{r}{pe}}}}{2\mu_p}                                                           \\
                           & =\sqrt{\frac{\mu_p}{\st{r}{pe}}}
\end{align*}

The maximum $\Delta V$ from a gravity assist occurs when the limiting velocity of the satellite is $v_\infty=\sqrt{\frac{\mu_p}{\st{r}{pe}}}$, and it results in an assist with $\Delta V = \sqrt{\frac{\mu_p}{\st{r}{pe}}}$ along the vector pointing from periapsis to the planet.

For the sake of curiosity, we shall now find (using Equation \eqref{Hyperbolic Flyby Eccentricity}) the eccentricity that this hyperbolic trajectory will have.

\begin{align*}
    e & = \frac{\st{r}{pe}v_\infty^2}{\mu_p}+1                   \\
      & = \frac{\st{r}{pe}(\sqrt{\mu_p/\st{r}{pe}})^2}{\mu_p}+1 \\
      & = \frac{\st{r}{pe}(\mu_p/\st{r}{pe})}{\mu_p}+1          \\
      & = \frac{\st{r}{pe}\mu_p}{\mu_p \st{r}{pe}}+1            \\
      & = 1+1                                                     \\
      & = 2
\end{align*}

The eccentricity to maximize $\Delta V$ of a gravity assist is $e=2$. Equation \eqref{Hyperbolic entry/exit angle} shows that this gives $\st{\theta}{hyp}=60^{\circ}=\frac{\pi}{3}$.

\subsubsection{Gravity Assist Conclusion}

The term ``gravity assist'' is possibly a bit of a misnomer, as the gravity serves only to redirect the velocity of the spacecraft, but does not change its magnitude. A spacecraft can be accelerated in its orbit by a planet if it passes behind the planet. Really what's happening is the spacecraft is using the planet's mass to redirect its speed, resulting in a change in the spacecraft's orbit. A well-timed flyby can act to change the velocity of the spacecraft in many different directions. From equation \eqref{Gravity Assist Delta V}, it can be seen that $\Delta V$ is maximized by minimizing $\st{r}{pe}$ and bringing $v_\infty$ as close as possible to $\sqrt{\frac{\mu_p}{\st{r}{pe}}}$. At this value for $v_\infty$, $\Delta V = \sqrt{\frac{\mu_p}{\st{r}{pe}}}$ and the hyperbola will have an eccentricity of $e=2$. It can also be seen that the greater the mass of the planet (and therefore $\mu$), the larger the effect will be. Another way of stating this is that to maximize the effect of a gravity assist, the hyperbolic trajectory should maximize its change in direction, which means minimizing its eccentricity (provided the eccentricity is greater than one). From Equation \eqref{Hyperbolic Flyby Eccentricity}, it can be seen that this entails minimizing $\st{r}{pe}$ for any given $v_\infty$.

\bigskip\bigskip
\subsection{Patched Conic Approximation}\label{sec:Patched Conics}

So far, the assumption has been made that a satellite has, at any moment, only one gravitational attractor. This leads to convenient results, such as conic section trajectories and simple analytical expressions for position, velocity, and the like. This approximation falls apart when dealing with orbits in which multiple bodies are referenced. The patched conic approximation is one in which you view each body as having its own ``sphere of influence'', and neglect the gravitational effect of all bodies aside from the one whose sphere of influence a spacecraft is in. Its name eludes to the patching together Keplarian trajectories about different bodies into a continuous set of conic sections.

To motivate this discussion of patched concis, we will examine a mission to Saturn. A spacecraft begins in a low parking orbit about Earth, and then performs a burn to exceed escape velocity. It then flies off from Earth, and enters an interplanetary trajectory about the sun. It orbits the sun for some fraction of its orbital period, after which it flies past Jupiter, getting a gravity assist. At this point, its orbit about the sun has changed, and it is on a trajectory Saturn. Once near Saturn, the spacecraft performs a capture burn to enter a circular parking orbit. This entire trajectory consisted of only two burns; an escape burn and a capture burn. This creates, theoretically, three distinct trajectories (initial parking orbit, transfer orbit, and final parking orbit).

While the actual mission consisted of only three trajectories, a total of 7 conic sections are needed to analyze the trajectory with Keplarian conic sections. The spacecraft's initial orbit about Earth is a circle (conic 1). After the escape burn it becomes hyperbolic (conic 2). Its trajectory around the sun is some elliptic trajectory (conic 3). As it encounters Jupiter, its trajectory from the Jupiterian frame of reference is hyperbolic (conic 4). It then enters another ellipse around the sun (conic 5). As it approaches Saturn, its trajectory about Saturn will be hyperbolic (conic 6). Finally, it circularizes into a parking orbit around Saturn (conic 7). Throughout this process, the reference frame was implicitly switched from Earth, to the sun, to Jupiter, back to the sun, and finally to Saturn.

The conic section shape of a trajectory comes from single-gravitational-attractor systems (note that Equation \eqref{Differential Equation} assumed a single gravitational attractor). To approximate the trajectory as a series of conic sections, all other gravitational influences aside from that of the primary reference must be neglected. Therefore, to approximate a trajectory as a series of patched conic sections, an approximation must be made that at any point there is only one gravitational influence on a spacecraft. Because the vast majority of the time in this interplanetary trajectory is spent under primarily the influence of the sun, with only brief interludes of Terran, Jupiterian, and Saturnian gravity, the approximation proves largely valid. The exact construction of spheres of influence varies by application and publication. The radius of a sphere of influence generally depends on the mass of a body, the mass of its parent body, and the separation of the two bodies.

While the patched conics approximation is only an approximation, and has its flaws, it it a necessary approximation in order to generate trajectories analytically. Much research has been done on the n-body problem (that is to say, the problem of finding trajectories under multiple gravitational influences), with the unfortunate conclusion being that it is not possible to calculate with analytical methods the trajectory of a spacecraft under more than one gravitational influence. The only alternative, therefore, to the patched conics approximation is to numerically integrate gravitational acceleration.

\bigskip\bigskip
\subsection{Conclusion}

A gravity assist is a method of using a hyperbolic flyby around a planet to change a satellite's velocity vector around the sun. If the limiting velocity of the hyperbolic orbit and the closest approach altitude are known, then the semi-major axis and eccentricity of the flyby can be calculated as
\[a=\frac{-\mu_p}{v_\infty^2}\qquad\text{and}\qquad e=\frac{\st{r}{pe}v_\infty^2}{\mu_p}+1\]

The magnitude of the change in velocity vector about the sun from a flyby is given by
\[\st{\Delta V}{assist}=\frac{2\mu_p v_\infty}{\st{r}{pe}v_\infty^2+\mu_p}\]

To optimize this $\Delta V$, the closest approach distance should be minimized. For any given closest approach distance,
\[v_{\infty,\text{best}}=\sqrt{\frac{\mu_p}{\st{r}{pe}}}=\st{v}{circ@pe}\]

Yielding a $\Delta V$ of
\[\st{v}{assist,best} = \sqrt{\frac{\mu_p}{\st{r}{pe}}}\]

The patched conic approximation allows trajectories to be calculated between multiple bodies easily, without requiring the use of numerical integration. While the approximation does not hold for all instances, it is generally a helpful tool for $\Delta V$ and trajectory approximation.

\end{document}